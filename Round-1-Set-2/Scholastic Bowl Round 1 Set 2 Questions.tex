\documentclass{report}
\usepackage{hyperref}
\usepackage{graphicx}
\usepackage{enumerate}
\usepackage{enumitem}
\usepackage{xcolor}
\usepackage{import}
\usepackage{pdfpages}
\usepackage{microtype}


\usepackage{geometry}
 \geometry{
 left=1 in,
 right=1 in,
 bottom=1 in,
 top=1 in
}

\setlength\parindent{0pt}

\definecolor{lightGray}{HTML}{d8dde6}

\newcommand*{\backtrack}{\setcounter{enumi}{\numexpr\theenumi-1\relax}}

\title{\textbf{Scholastic Bowl} \\ Round 1 --- Set 2}
\author{Pranaav Sureshkumar \\ \href{mailto:pranaav2@illinois.edu}{pranaav2@illinois.edu}}
\date{\today}

\begin{document}

\includepdf[pages=-]{C:/Users/prana/Documents/GitHub/Scholastic-Bowl-Questions-1/Title Pages/Round 1 Set 2 Title Page.pdf}

\maketitle

\import{C:/Users/prana/Documents/GitHub/Scholastic-Bowl-Questions-1/Rules/}{Rules.tex}

\newpage

\vspace*{\fill}
\centering
\thispagestyle{empty}
\Large
Questions begin on the next page.
\vspace*{\fill}

\normalsize
\newpage
\setcounter{page}{1}

\begin{enumerate}
    % Question 1
    \item \textbf{MATH} \\ \textit{These} processes are called greedy if they always make the locally optimal choice. Turing machines can execute any of \textit{these} processes. Big O notation (*) is used to describe the time and space complexity of \textit{these} processes. For 10 points, name \textit{these} finite, step-by-step sequences of unambiguous instructions that can be used to perform a computation. \\ ANSWER: \underline{algorithm} [prompt on \textit{program}] \backtrack
    \item \textbf{MISCELLANEOUS} \\ Name \textit{these} 1990s bands with prominent drummers. For 10 points each,
    \begin{enumerate}[label=\Alph*]
        \item Dave Grohl drummed on hits like \textit{Heart-Shaped Box} and \textit{Smells Like Teen Spirit} for \textit{this} band. He went on to form the Foo Fighters after the suicide of \textit{this} band’s lead singer, Kurt Cobain. \\ ANSWER: \underline{Nirvana}
        \item Tré Cool replaced John Kiffmeyer in 1990 as \textit{this} band's drummer. He drummed in hits like \textit{Brain Stew}, \textit{American Idiot}, and \textit{Know Your Enemy}. \\ ANSWER: \underline{Green Day}
        \item Chad Smith drummed for \textit{this} band since 1988, producing hits such as \textit{Snow}, \textit{Under the Bridge}, and \textit{Californication}. \\ ANSWER: \underline{Red Hot Chili Peppers}
    \end{enumerate}

    % Question 2
    \item \textbf{SOCIAL SCIENCE} \\ Plato once referred to \textit{these} animals as ``the only featherless bipeds with broad, flat nails (*).'' A philosophical stance named after \textit{these} animals was pioneered by Petrarch. Friedrich Nietzsche \colorbox{lightGray}{[\textbf{Neet}\textperiodcentered shuh]} claimed that \textit{these} animals killed God. For 10 points, name \textit{these} animals which were often the subject of Italian Renaissance art like the \textit{Mona Lisa}. \\ ANSWER: \underline{human}s \backtrack
    \item \textbf{SCIENCE} \\ In mechanical physics, almost all translation quantities have rotational analogues. Name the following quantities and operations used in rotational mechanics. For 10 points each,
    \begin{enumerate}[label=\Alph*]
        \item \textit{This} quantity is the rotational analogue of force. When a force is applied perpendicular to a lever arm, \textit{this} quantity is the force applied multiplied by the distance to the pivot. \\ ANSWER: \underline{torque}
        \item Torque is the rotational analogue of force. Therefore, by Newton's Second Law, it is also equal to angular acceleration multiplied by \textit{this} quantity. \\ ANSWER: \underline{rotational inertia} [also accept \underline{moment of inertia}, \underline{angular mass}, \underline{rotational mass}, or \underline{second moment of mass}]
        \item When the force and lever arm vectors are not perpendicular, \textit{this} vector operation must be used to find torque. Unlike the dot product, this operation yields a vector. \\ ANSWER: \underline{cross} product
    \end{enumerate} 

    % Question 3
    \item \textbf{HISTORY} \\ During the War of 1812, \textit{this} poet watched from a British ship as the American troops in Fort McHenry raised their flag (*). Abolitionists criticized \textit{this} poet's words that America was the land of the free\textemdash instead referring to it as ``the land of the free and the home of the oppressed.'' On March 26, 2024, a Baltimore bridge named after \textit{this} poet collapsed. For 10 points, name \textit{this} poet who wrote \textit{The Star-Spangled Banner}, which later became the US national anthem. \\ ANSWER: Francis Scott \underline{Key} \backtrack
    \item \textbf{MISCELLANEOUS} \\ In French culinary theory, there are \textit{mother} sauces from which many other \textit{daughter} sauces are derived. Name the following mother sauces by the daughter sauces they're used to make. For 10 points each,
    \begin{enumerate}[label=\Alph*]
        \item \textit{This} mother sauce is the base for salsas and many pasta dishes. \\ ANSWER: \underline{tomato} sauce
        \item \textit{This} mother sauce is the base for tartar sauce and ranch dressing. \\ ANSWER: \underline{mayo}nnaise
        \item \textit{This} mother sauce is the base for foyot \colorbox{lightGray}{[foe\textperiodcentered \textbf{yote}]} sauce and is often served with Eggs Benedict. \\ ANSWER: \underline{hollandaise} sauce
    \end{enumerate}

    % Question 4
    \item \textbf{LITERATURE} \\ In \textit{this} cautionary tale, Oceania, Eurasia, and Eastasia are constantly at war, which Emmanuel Goldstein believes is the only way to maintain peace. In \textit{this} novel, Winston Smith is betrayed and captured by secret members of the Thought Police (*) who turn him in to the Party. Four government agencies in \textit{this} novel are ironically named the ministries of truth, peace, love, and plenty. For 10 points, name \textit{this} dystopian novel about totalitarianism, written by George Orwell, which popularized terms such as \textit{thoughtcrime}, \textit{doublethink}, and \textit{Big Brother}. \\ ANSWER: \underline{Nineteen Eighty-Four} \backtrack
    \item \textbf{SCIENCE} \\ Every quantity you can think of can be expressed in terms of SI units. Name the following quantities by their definition in SI units. For example, if I said \textit{meters per second}, you would answer \textit{speed} or \textit{velocity}\textemdash both are acceptable. For 10 points each,
    \begin{enumerate}[label=\Alph*]
        \item \textit{This} quantity is measured in meters cubed. \\ ANSWER: \underline{volume} [do \textbf{not} accept or prompt on \textit{liter}]
        \item \textit{This} quantity is measured in ampere-seconds. \\ ANSWER: electric \underline{charge} [do \textbf{not} accept or prompt on \textit{coulomb}]
        \item \textit{This} quantity is measured in kilogram-meters squared per second squared. \\ ANSWER: \underline{energy} [also accept \underline{work} or \underline{torque} or \underline{heat} (but \textbf{not} \textit{temperature}); do \textbf{not} accept or prompt on \textit{joule}]
    \end{enumerate} 

    % Question 4
    \item \textbf{HISTORY}

\end{enumerate}


\vspace*{0.5 cm}
\centering
\rule{10 cm}{0.4pt}

\Large
END OF QUESTION SET
\newpage

\vspace*{\fill}
\centering
\thispagestyle{empty}
\Large
No questions on this page
\vspace*{\fill}

\newpage

\vspace*{\fill}
\centering
\thispagestyle{empty}
\Large
No questions on this page
\vspace*{\fill}

\end{document}