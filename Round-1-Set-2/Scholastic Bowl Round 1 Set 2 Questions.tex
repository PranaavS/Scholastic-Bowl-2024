\documentclass{report}
\usepackage{hyperref}
\usepackage{graphicx}
\usepackage{enumerate}
\usepackage{enumitem}
\usepackage{xcolor}
\usepackage{import}
\usepackage{pdfpages}


\usepackage{geometry}
 \geometry{
 left=1 in,
 right=1 in,
 bottom=1 in,
 top=1 in
}

\setlength\parindent{0pt}

\definecolor{lightGray}{HTML}{d8dde6}

\newcommand*{\backtrack}{\setcounter{enumi}{\numexpr\theenumi-1\relax}}

\title{\textbf{Scholastic Bowl} \\ Round 1 --- Set 2}
\author{Pranaav Sureshkumar \\ \href{mailto:pranaav2@illinois.edu}{pranaav2@illinois.edu}}
\date{\today}

\begin{document}

\includepdf[pages=-]{C:/Users/prana/Documents/GitHub/Scholastic-Bowl-Questions-1/Title Pages/Round 1 Set 2 Title Page.pdf}

\maketitle

\import{C:/Users/prana/Documents/GitHub/Scholastic-Bowl-Questions-1/Rules/}{Rules.tex}

\newpage

\vspace*{\fill}
\centering
\thispagestyle{empty}
\Large
Questions begin on the next page.
\vspace*{\fill}

\normalsize
\newpage
\setcounter{page}{1}

\begin{enumerate}
    % Question 1
    \item \textbf{MATH} \\ \textit{These} processes are called greedy if they always make the locally optimal choice. Turing machines can execute any of \textit{these} processes. Big O notation is used to describe the time and space complexity of \textit{these} processes. For 10 points, name \textit{these} step-by-step sequences of unambiguous instructions that can be used to perform a computation. \\ ANSWER: \underline{algorithm} [prompt on \textit{program}] \backtrack
\end{enumerate}


\vspace*{0.5 cm}
\centering
\rule{10 cm}{0.4pt}

\Large
END OF QUESTION SET
\newpage

\vspace*{\fill}
\centering
\thispagestyle{empty}
\Large
No questions on this page
\vspace*{\fill}

\newpage

\vspace*{\fill}
\centering
\thispagestyle{empty}
\Large
No questions on this page
\vspace*{\fill}

\end{document}