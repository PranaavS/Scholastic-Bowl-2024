\documentclass{report}
\usepackage{hyperref}
\usepackage{graphicx}
\usepackage{enumerate}
\usepackage{enumitem}
\usepackage{xcolor}
\usepackage{import}
\usepackage{pdfpages}


\usepackage{geometry}
 \geometry{
 left=1 in,
 right=1 in,
 bottom=1 in,
 top=1 in
}

\setlength\parindent{0pt}

\definecolor{lightGray}{HTML}{d8dde6}

\newcommand*{\backtrack}{\setcounter{enumi}{\numexpr\theenumi-1\relax}}

\title{\textbf{Scholastic Bowl} \\ Round 1 --- Set 1}
\author{Pranaav Sureshkumar \\ \href{mailto:pranaav2@illinois.edu}{pranaav2@illinois.edu}}
\date{\today}

\begin{document}

\includepdf[pages=-]{C:/Users/prana/Documents/GitHub/Scholastic-Bowl-Questions-1/Title Pages/Round 1 Set 1 Title Page.pdf}

\maketitle

\import{C:/Users/prana/Documents/GitHub/Scholastic-Bowl-Questions-1/Rules/}{Rules.tex}

\newpage

\vspace*{\fill}
\centering
\thispagestyle{empty}
\Large
Questions begin on the next page.
\vspace*{\fill}

\normalsize
\newpage
\setcounter{page}{1}

\begin{enumerate}

    % Question 1
    \item \textbf{HISTORY} \\ As the first absolutist monarch of England, \textit{this} king ruled by divine right, challenging papal supremacy. After being refused a divorce, \textit{this} king founded the Anglican Church (*). The Broadway musical SIX reimagines \textit{this} king’s wives as modern-day pop singers. A jingle to help students remember the fate of \textit{this} king’s six wives goes: \textit{“Divorced, beheaded, died; divorced, beheaded, survived.”} For 10 points, name \textit{this} second Tudor monarch. \\ ANSWER: \underline{Henry VIII} [prompt on \textit{Henry} or \textit{Henry Tudor}] \backtrack
    \item \textbf{SCIENCE} \\ A bowling ball is attached to one end of a string, and the other end is attached to the ceiling, forming a pendulum. The bowling ball is then pulled back to a height of five meters above the resting position. Air resistance is negligible. The ball is released from rest. For 10 points each,
    \begin{enumerate}[label=\Alph*]
        \item Although the ball swings quickly through its path, it will never rise to a height greater than five meters because \textit{this} is always conserved in a closed system. \\ ANSWER: \underline{energy} [this word only; do \textbf{not} accept or prompt on \textit{kinetic energy} or \textit{potential energy}]
        \item Just before it is released, the ball has only \textit{this} form of energy. \\ ANSWER: gravitational \underline{potential} energy
        \item The ball moves fastest at the bottom of its path because \textit{this} type of energy is maximized there. \\ ANSWER: \underline{kinetic} energy
    \end{enumerate}
    
    % Question 2
    \item \textbf{LANGUAGE ARTS} \\ \textit{These} statements can be either analytical, explanatory, or argumentative. \textit{These} statements are considered weak if they are imprecise, are not debatable, or are too broad. Dr. Gary Thomas referred to \textit{these} statements as “your argument in a nutshell (*).” For 10 points, name \textit{these} statements which are usually placed at the end of the introduction to an essay. \\ 
    ANSWER: \underline{thesis} \backtrack
    \item \textbf{MATH} \\ Name the following commonly used programming keywords by their definitions or uses. For 10 points each,
    \begin{enumerate}[label=\Alph*]
        \item \textit{This} keyword is usually placed at the end of a function. Although there may be multiple of \textit{these} keywords in a function, only one such statement can be executed in a single call. \\ ANSWER: \underline{return} [prompt on \textit{yield} until ``one'' is read]
        \item If a function does not return anything, \textit{this} keyword is put in its header. \\ ANSWER: \underline{void}
        \item In statically typed programming languages, \textit{this} keyword is put just before a variable’s name to indicate that the variable is either true or false. \\ ANSWER: \underline{bool}ean
    \end{enumerate}

    % Question 3
    \item \textbf{LITERATURE} \\ \textit{This} Greek demigod is the son of Zeus and Alcmene \colorbox{lightGray}{[Alk\textperiodcentered \textbf{mee}\textperiodcentered nee]}. When Hera sent serpents to kill \textit{this} demigod in his cradle, he strangled them both. In a manic state induced by Hera, \textit{this} demigod killed his wife and children (*). As punishment for his murders, \textit{this} demigod was forced to serve a king for twelve years. For 10 points, name \textit{this} demigod who slayed the Hydra as one of his twelve labors. \\ ANSWER: \underline{Hercules} [also accept \underline{Heracles}] \backtrack \newpage
    \item \textbf{SCIENCE} \\ Every quantity you can think of can be expressed in terms of SI units. Name the following quantities by their definition in SI units. For example, if I said \textit{meters per second}, you would answer \textit{speed} or \textit{velocity}\textemdash both are acceptable. For 10 points each,
    \begin{enumerate}[label=\Alph*]
        \item \textit{This} quantity is measured in inverse seconds. \\ ANSWER: \underline{frequency} [do \textbf{not} accept or prompt on \textit{hertz}]
        \item \textit{This} quantity is measured in ampere-seconds. \\ ANSWER: \underline{charge} [do \textbf{not} accept or prompt on \textit{coulomb}]
        \item \textit{This} quantity is measured in kilogram-meters per second squared. \\ ANSWER: \underline{force} [do \textbf{not} accept or prompt on \textit{newton}]
    \end{enumerate}

    % Question 4
    \item \textbf{SOCIAL SCIENCE} \\ Prophets and messengers of \textit{this} religion include Adam, Noah, Abraham, Moses, and Jesus. Shia and Sunni \colorbox{lightGray}{[\textbf{Soo}\textperiodcentered nee]} (*) are the two major denominations of \textit{this} religion. Adherents of \textit{this} religion undergo a \textit{hajj} to Mecca as one of \textit{this} religions five pillars. For 10 points, name \textit{this} second-largest religion whose holy text is the Quran. \\ ANSWER: \underline{Islam} \backtrack
    \item \textbf{MISCELLANEOUS} \\ When you go shopping, you may feel like you have a large selection of items to choose from. However, many household brands are actually consolidated under one \textit{umbrella} company. Name each of the following umbrella companies by the brands they own. For 10 points each,
    \begin{enumerate}[label=\Alph*]
        \item \textit{This} umbrella company owns Tic-Tac, Nutella, and Butterfinger. \\ ANSWER: \underline{Ferrero} Group
        \item \textit{This} umbrella company owns Frito Lay, Lipton, and Gatorade. \\ ANSWER: \underline{Pepsi}Co
        \item \textit{This} umbrella company owns National Geographic, ESPN, and Marvel. \\ ANSWER: Walt \underline{Disney}
    \end{enumerate}

    % Question 5
    \item \textbf{SCIENCE} \\ Real-world examples of \textit{these} objects have a nonzero mass, so their rotational inertia is their mass multiplied by the square of their radius. When \textit{these} objects have zero mass, as in the theoretical Atwood machine, they are (*) considered \textit{ideal}. For 10 points, name \textit{this} machine that derives its mechanical advantage by reversing the direction of input force with a rope suspended over a wheel. \\ ANSWER: \underline{pulley} \backtrack
    \item \textbf{SOCIAL SCIENCE} \\ Zeno of Elea was famous for posing \textit{this} type of thought experiment. For 10 points each,
    \begin{enumerate}[label=\Alph*]
        \item Name \textit{these} statements in which two pieces of seemingly sound logic lead to a contradiction. \\ ANSWER: \underline{paradox}
        \item Many paradoxes can be resolved by applying \textit{this} principle which states that the simplest explanation is most often correct. \\ ANSWER: \underline{Occam's razor}
        \item Despite being used to resolve most paradoxes, Occam's razor is said to be ``dulled'' by \textit{this} philosopher's \textit{beard}. Name this Ancient Greek philosopher who taught Aristotle and was taught by Socrates. \\ ANSWER: \underline{Plato}
    \end{enumerate}
    \newpage

    % Question 6
    \item \textbf{MISCELLANEOUS} \\ Metheglin \colorbox{lightGray}{[\textbf{meh}\textperiodcentered thuh\textperiodcentered gluhn]} and melomel are two varieties of this beverage. In Norse mythology, \textit{this} beverage is said to be crafted from the blood of Kvasir \colorbox{lightGray}{[\textbf{Kwa}\textperiodcentered sear]} and to turn those who drink it into a poet or scholar. For 10 points, name \textit{this} alcoholic beverage similar to wine but fermented with honey.\\ ANSWER: \underline{mead} \backtrack
    \item \textbf{HISTORY} \\ Name these prominent historical figures who went blind. For 10 points each,
    \begin{enumerate}[label=\Alph*]
        \item \textit{This} American soul musician composed \textit{Georgia on My Mind}, \textit{Hit the Road Jack}, and \textit{You Don't Know Me}. \\ ANSWER: Ray \underline{Charles}.
        \item The most prestigious award for journalism is named after \textit{this} New York newspaper publisher. \\ ANSWER: Joseph \underline{Pulitzer}
        \item \textit{This} impressionist is known best for his numerous paintings of water lilies. \\ ANSWER: Claude \underline{Monet}
    \end{enumerate}

    % Question 7
    \item \textbf{HISTORY} \\ In 1920, ten billion of \textit{these} invasive animals devastated crops in Australia. Control methods for \textit{these} animals include shooting, poisoning, fumigating, and dismembering. In 1807, Napoleon was famously attacked by \textit{these} animals and forced to flee (*). For 10 points, name \textit{these} animals whose feet are thought to be lucky.\\ ANSWER: \underline{rabbit} [also accept \underline{bunny} or \underline{hare}] \backtrack
    \item \textbf{SCIENCE} \\ Name these structures used in bridges. For 10 points each,
    \begin{enumerate}[label=\Alph*]
        \item \textit{This} structure is an assembly of beams, connected by \textit{nodes} that creates a bridge's rigid structure. \\ ANSWER: \underline{truss}
        \item \textit{These} structures, which serve a similar function to pillars, extend from water to the underside of the bridge. \\ ANSWER: \underline{pier}
        \item \textit{These} structures, which are named after aquatic animals, are used to protect bridges from colliding with a ship. \\ ANSWER: \underline{dolphin}
    \end{enumerate}

    % Question 8
    \item \textbf{MATH} \\ \textit{This} object is formally defined as the \textit{image} of an interval to a topological space by a continuous function. \textit{This} object can be  \textit{fractal} like the one named after Bill Gosper or \textit{space-filling} like the one named after David Hilbert. \textit{This} object can be embedded in any number of dimensions, but its manifold dimension is always one, unlike a surface (*). For 10 points, name \textit{this} mathematical object that is similar to a line but which does not have to be straight. \\ ANSWER: \underline{curve} \backtrack
    \item \textbf{HISTORY} \\ Rome wasn't built in a day. In fact, modern historians aren't sure \textit{how} exactly Rome was founded\textemdash even ancient Romans weren't sure. For 10 points each,
    \begin{enumerate}[label=\Alph*]
        \item One popular legend points to two brothers who jointly founded Rome. Name either. \\ ANSWER: \underline{Remus} [also accept \underline{Romulus}]
        \item After being left to die on the bank of a river, the legends claim that both brothers were found and raised by \textit{this} animal. \\ ANSWER: she-\underline{wolf}
        \item After discovering their royal heritage, both brothers agree to found a city of their own, but they cannot agree on which hill to build the city upon. After a long dispute, Romulus does \textit{this} to Remus and becomes the first king of Rome. \\ ANSWER: \underline{murder} [also accept equivalent answers]
    \end{enumerate}
    
    % Question 9
    \item \textbf{MISCELLANEOUS} \\ \textit{This} quanitiy is a measure of the ethanol concentration of a beverage. A degree symbol is commonly used to abbreviate \textit{this} quantity. In the US, \textit{this} quanity ranges from zero for a non-alcoholic beverage to two hundred for pure ethanol (*). For 10 points, name \textit{this} quantity that is exactly double a drink's ABV. \\ ANSWER: \underline{proof} \backtrack
    \item \textbf{LANGUAGE ARTS} \\ Consider the following sentence: \textit{The man was six feet tall, but a dog jumped over him all the same.} For 10 points each,
    \begin{enumerate}[label=\Alph*]
        \item The word ``the'' takes \textit{this} part of speech in the sentence. ``The'' is the \textit{definite} form of \textit{this} part of speech, unlike the word ``a''.\\ ANSWER: \underline{article} [also accept \underline{adjective} or \underline{adverb}]
        \item The word ``him'' takes \textit{this} part of speech in the sentence. \\ ANSWER: \underline{pronoun}
        \item When a pronoun refers to a noun, the noun is called \textit{this}. In the sentence, the word ``man'' serves as \textit{this} type of noun. \\ ANSWER: \underline{antecedent}
    \end{enumerate}

    % Question 10
    \item \textbf{SCIENCE} \\ \textit{This} organelle used to be a free-living prokaryote and is thought to have become part of eukaryotic cells by endosymbiosis. \textit{This} organelle has a double membrane and uses aerobic respiration to generate ATP for its cell. The Krebs cycle takes place in the matrix of \textit{this} organelle (*). For 10 points, name \textit{this} organelle which is commonly referred to as ``the powerhouse of the cell.'' \\ ANSWER: \underline{mitochondrion} \backtrack
    \item \textbf{SOCIAL SCIENCE} \\ World War One was caused by a plethora of factors. For 10 points each,
    \begin{enumerate}[label=\Alph*]
        \item The acronym MAIN is used to help students remember the four main causes of World War One. The letters stand for militarism, alliances, imperialism, and \textit{this} fourth cause which is defined as pride for one's own country. \\ ANSWER: \underline{nationalism}
        \item The four previous causes are broad and long-term. The immediate cause of World War One was Gavrilo Princip's assassination of \textit{this} Archduke. \\ ANSWER: Archduke \underline{Franz Ferdinand}
        \item \textit{This} German Chancellor used \textit{realpolitik} to weave the elaborate alliance system that held Europe together before World War One. When \textit{he} was fired by William II, the alliance system crumbled, which was a major cause of World War One. \\ ANSWER: Otto von \underline{Bismarck}
    \end{enumerate}
\end{enumerate}

\vspace*{0.5 cm}
\centering
\rule{10 cm}{0.4pt}

\Large
END OF QUESTION SET
\newpage

\vspace*{\fill}
\centering
\thispagestyle{empty}
\Large
No questions on this page
\vspace*{\fill}

\newpage

\vspace*{\fill}
\centering
\thispagestyle{empty}
\Large
No questions on this page
\vspace*{\fill}

\end{document}