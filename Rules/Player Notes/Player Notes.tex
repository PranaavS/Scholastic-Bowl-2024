\documentclass{report}
\usepackage{hyperref}
\usepackage{graphicx}
\usepackage{enumerate}
\usepackage{enumitem}
\usepackage{xcolor}
\usepackage{import}
\usepackage{pdfpages}
\usepackage{microtype}
\usepackage[framemethod=TikZ]{mdframed}

\mdfdefinestyle{sample}{%
    linecolor=gray,
    outerlinewidth=1pt,
    roundcorner=0pt,
    innertopmargin=\baselineskip,
    innerbottommargin=\baselineskip,
    innerrightmargin=20pt,
    innerleftmargin=20pt,
    backgroundcolor=white}



\usepackage{geometry}
 \geometry{
 left=1 in,
 right=1 in,
 bottom=1 in,
 top=1 in
}

\setlength\parindent{0pt}

\definecolor{lightGray}{HTML}{d8dde6}

\newcommand*{\backtrack}{\setcounter{enumi}{\numexpr\theenumi-1\relax}}

\title{\textbf{Scholastic Bowl} \\ Round 1 --- Set 2}
\author{Pranaav Sureshkumar \\ \href{mailto:pranaav2@illinois.edu}{pranaav2@illinois.edu}}
\date{\today}

\begin{document}

\thispagestyle{empty}
The following are notes for competitors. Please note that your moderator may override these rules when they are instructed to do so.

\vspace*{0.3 cm}
On reading
\begin{itemize}
    \item Competitors may stop the moderator to request that the moderator speak louder or more clearly.
\end{itemize}

\vspace*{0.3 cm}
On answers
\begin{itemize}
    \item The bolded part of each answer is required to receive points. Other parts are optional.
    \item Brackets beside answers provide the moderator notes and additional information to assist with scoring (which may override the rules provided on this page).
    \item Unless the answer is a proper noun, the moderator must accept both singular and plural variants of an answer. 
    \item ``prompt'': The moderator will say \textit{prompt} or indicate in some other way that additional or alternate information is required. The moderator may prompt multiple times on the same question as long as each subsequent answer provides more information and each answer is correct.
\end{itemize}

\vspace*{0.3 cm}
On powerplays
\begin{itemize}
    \item All toss-up questions contain the symbol \underline{(*)} which indicates the \textit{powerplay} cutoff.
    \item When a toss-up is answered before the powerplay cutoff, the team which answered receives 15 (instead of 10) points for that toss-up.
\end{itemize}

\vspace*{0.3 cm}
A sample toss-up question is shown below.

\begin{mdframed}[style=sample]
    \begin{enumerate}
        \setcounter{enumi}{7}
        \item \textbf{SCIENCE} \\ The release of energy in \textit{these} events is described by elastic-rebound theory. \textit{These} events are detected by triangulation of P and S waves. The modified Mercalli scale measures the (*) intensity of \textit{these} events, unlike a logarithmic scale for their magnitude, which is named for Charles Richter. \textit{These} events begin at an epicenter and can cause tsunamis. For 10 points, name \textit{these} natural disasters which are caused by rupturing faults in the crust. \\ ANSWER: \underline{earthquake}
    \end{enumerate}
    \end{mdframed}

\vspace*{0.3 cm}
On timing
\begin{itemize}
    \item If the first team to buzz in answers a bonus incorrectly and \textit{more} than \textbf{five} seconds remain, the other team receives the remaining time to buzz in.
    \item If the first team to buzz in answers a bonus incorrectly and \textit{fewer} than \textbf{five} seconds remain, the other team receives \textbf{five} seconds to buzz in.
    \item If a toss-up question requires compuation, it will be prefaced with ``pencils and paper ready.''
    \item If a bonus question requires compuation, only the introductory sentence will be prefaced with ``pencils and paper ready.''
    \item For each question, teams receive the amount of time indicated in the table below to buzz in.

\end{itemize}

\vspace*{\fill}

\begin{center}
\begin{tabular}{||c|c||} 
    \hline
    Type & Time (in seconds) \\
    \hline\hline
    Toss-up & $10$ \\ 
    \hline
    Toss-up with computation & $30$ \\ 
    \hline
    Bonus & $10\times 3$ questions \\ 
    \hline
    Bonus with computation & $10\times 3$ questions \\
    \hline
\end{tabular}
\end{center}




\end{document}